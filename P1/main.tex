\chapter{\label{ch:p1} Pràctica P1 -- Introducció a Eclipse}

\section{Objectius}

En aquesta primera pràctica es vol familiaritzar l'estudiant amb la placa,
l'estructura del projecte i el codi sobre el que es treballarà, l'us de l'entorn de
desenvolupament basat en Eclipse, i també la metodologia general de les
pràctiques.

\section{Desenvolupament}


\subsection{Verificació de la placa}

Es comença explicant la necessitat i el procés de verificar la placa,
i la diferència entre carregar codi a la RAM i fer-ho a la Flash. Com que
el procés ve precarregat a la placa, s'enten que forma part del material i
per tant no s'explicarà aquí.

S'executen les diferents proves per comprovar que la placa es troba en bon estat.

Llavors s'introdueix l'entorn de desenvolupament basat en Eclipse. Es donen
les instruccions per començar-hi a treballar. Com que s'ha fet servir un entorn
diferent pel desenvolupament (basat en Netbeans i en un altre sistema operatiu),
no s'entrarà massa en detall.

S'obre el Workspace a Eclipse. S'obre el fitxer \filename{main.c} a l'editor.
Es compila el projecte i es verifica que s'ha generat el fitxer \filename{build/ch.elf}.


\subsection{Càrrega del projecte a la RAM de la placa}

El projecte exemple que es dona conté un codi a \filename{main.c} que només encen
i apaga contínuament el LED verd de la placa.

\begin{minted}{c}
#include "Base.h"     // Basic definitions

// Function that blinks the green LED

void ledBlink(void) {
    while (1) {
        // Turn on the green LED using the BSRR register
        (LEDS_PORT->BSRR.H.set) = BIT(GREEN_LED_PAD);

        // Wait 200ms
        SLEEP_MS(200);

        // Turn off the green LED using the BSRR register
        (LEDS_PORT->BSRR.H.clear) = BIT(GREEN_LED_PAD);

        // Wait 200ms
        SLEEP_MS(200);
    }
}

int main(void) {
    // Basic initializations
    baseInit();

    // Call the LED blink function
    // This function never returns
    ledBlink();

    // Return so that the compiler doesn't complain
    // It is not really needed as ledBlink never returns
    return 0;
}
\end{minted}
\vskip -1em

En primer lloc s'explica com carregar
el codi compilat a la RAM de la placa. Hem d'iniciar OpenOCD al menú d'aplicacions
externes, i un cop aquest hagi establert la connexió amb la placa, es carrega el codi
seleccionant la opció \textsf{Practica (RAM Load and Run)} del menú de debug.

Un cop el codi s'ha carregat, s'obre una sessió de depuració. El codi es troba parat sobre
la primera instrucció del programa (a la funció \fname{main}), i podem reprendre l'execució
fent clic al botó \textsf{Resume} del depurador. Ara el LED verd hauria d'encendre's i apagar-se
continuament. Es demana que parem el programa (i la sessió de depuració) fent clic a \textsf{Terminate}.


\subsection{Depuració del projecte}

Es continua explicant el funcionament del depurador, es demana que es carregui de nou
el codi i en lloc de polsar \textsf{Resume}, es faci servir \textsf{Step Into}, \textsf{Step Over}, etc.
per familiaritzar-se amb el fluxe d'execució.

A continuació s'explica com veure el contingut
de les variables i posar breakpoints. A mode d'exemple, es proposa afegir una variable global
\mintinline{c}|volatile uint32_t i=0;| i incrementar-la dins del bucle. Llavors, posar-li un
breakpoint i veure com s'interromp el flux d'execució en cada iteració / cicle del LED.
També observar el valor de la variable.

\textbf{Nota:} No es va aconseguir que funcionés la vista de variables globals, sempre que
s'afegia la variable aquesta sortia sense valor assignat. De tota forma, sí es podia afegir
un \emph{watch} amb el nom de la variable i d'aquesta forma veure el seu valor.


\subsection{Descripció del funcionament del programa}

En aquest apartat s'explica la funció de cadascun dels fitxers del projecte. S'expliquen les variables
del \filename{Makefile}, en concret la variable \mintinline{text}|PCSRC|, que especifica la llista
de fitxers de codi font a compilar i incloure a l'executable final.

També s'explica la funció de \filename{mcuconf.h}, \filename{halconf.h} i \filename{chconf.h},
que especifiquen la configuració del sistema operatiu, tant opcions generals com ajustos concrets
per al microcontrolador que fem servir (STM32F407).

S'explica la funció de \filename{Base.c} i \filename{Base.h}, els macros i les utilitats que contenen,
entre ells cal destacar \mintinline{c}|SLEEP_MS(x)| i \mintinline{c}|DELAY_US(x)|, que bloquegen
l'execució durant un nombre de mil·lisegons o microsegons, i \mintinline{c}|BIT(x)| i les constants
\mintinline{c}|BITx|, que evaluen a un nombre amb el bit indicat actiu i la resta a zero.

S'explica la funció de \filename{labBoard12.h}, que conté la descripció del hardware de la placa,
com per exemple els ports i els pins on estan connectats diferents perifèrics, entre ells els LEDs.

S'explica per sobre l'estructura del codi de \filename{main.c} i l'estructura de les línies GPIO
i com s'accedeixen.


\subsection{Modificació del programa}

Ara es demana, com a exercici bàsic, que es modifiqui \filename{main.c} per afegir una nova funció
\fname{ledSequence} que encengui els LEDs seguint la seqüència següent de manera repetiva, deixant
\SI{0.5}{\second} entre estat i estat:

\begin{center}
\ledpattern{X}{ }{ }{ } $\rightarrow$
\ledpattern{ }{X}{ }{ } $\rightarrow$
\ledpattern{ }{ }{X}{ } $\rightarrow$
\ledpattern{ }{ }{ }{X}
\end{center}

La funció s'implementa com segueix:

\begin{minted}{c}
#define TURN_ON(bits) (LEDS_PORT->BSRR.H.set) = (bits);
#define TURN_OFF(bits) (LEDS_PORT->BSRR.H.clear) = (bits);
#define TURN_ON_DURING(time, bits) \
        TURN_ON(bits); SLEEP_MS(time); TURN_OFF(bits);

void ledSequence(void) {
    while (1) {
        TURN_ON_DURING(500, GREEN_LED_BIT);
        TURN_ON_DURING(500, ORANGE_LED_BIT);
        TURN_ON_DURING(500, RED_LED_BIT);
        TURN_ON_DURING(500, BLUE_LED_BIT);
    }
}
\end{minted}
\vskip -1em

Llavors es crida des de \fname{main}:

\begin{minted}{diff}
     // Call the LED blink function
     // This function never returns
-    ledBlink();
+    //ledBlink();
+
+    // Call the LED sequence function
+    // This function never returns
+    ledSequence();
 
     // Return so that the compiler doesn't complain
\end{minted}
\vskip -1em

El programa es carrega a la placa i es comprova que té el funcionament esperat.


\subsection{Eines addicionals de depuració}

Per últim s'expliquen altres eines com ara: la finestra de ChibiOS/RT i la informació que ens proporciona,
la pestanya de memòria, com canviar de consola i la finestra \textsf{EmbSys Registers} que ens permet
veure i modificar l'estat dels registres del microcontrolador i com configurar-la.


\subsection{Stack Frames}

També s'explica què són els \emph{stack frames} i la informació que donen quan estem depurant.


\section{Conclusió}

La pràctica s'ha realitzat sense problema, excepte per la part de veure les variables.

\section{Ajustaments posteriors}

Cap canvi a destacar.

