\documentclass[catalan,parskip=half*,oneside,DIV=11,hidelinks]{scrreprt}

% encoding
\usepackage[utf8]{inputenc}
\usepackage[T1]{fontenc}
\usepackage{lmodern}
\usepackage{babel}

% formatting and fixes
\frenchspacing
\usepackage[style=spanish]{csquotes}
\MakeAutoQuote{«}{»}
\usepackage{bookmark}
\usepackage{scrhack}

% ADD ANY SPECIFIC PACKAGES HERE
% (CHEMISTRY, CODE, PUBLISHING)
\usepackage{multicol}
\usepackage{environ}
\usepackage{enumitem}
\usepackage[usenames,dvipsnames,svgnames,table]{xcolor}
\usepackage{tcolorbox}
\usepackage{adjustbox}
\usepackage{booktabs}
\usepackage[figure]{hypcap}

\usepackage{mathtools}
\usepackage{commath}
\usepackage{nicefrac}
\usepackage{siunitx}

\usepackage[american]{circuitikz}
\usetikzlibrary{calc}
\usetikzlibrary{arrows.meta}
\usetikzlibrary{automata}
\usepackage{tikz-timing}
\usepackage{minted}

% hyperlink setup / metadata
\usepackage{hyperref}
\AfterPreamble{\hypersetup{
  pdfauthor={Alba Méndez},
  pdftitle={Memòria final DSBM},
  pdfsubject={DSBM QT2018},
  pdfcreator={}, pdfproducer={},
  linkcolor={red!50!black},
  citecolor={blue!50!black},
  urlcolor={blue!80!black},
  bookmarksopen=true,
}}
\setlength{\footskip}{2.0cm}

% custom commands: BDF schematic and BSF symbol
\newcommand{\nodenamebit}[1]{\text{#1}}
\newcommand{\nodenamesingle}[2]{\text{#1}_{#2}}
\newcommand{\nodenamerange}[3]{\text{#1}_{#2..#3}}

% setup C listing appearance
\definecolor{bg}{rgb}{0.95,0.95,0.95}
\usemintedstyle{lovelace}
\setminted{
  breaklines, breakindent=1em, breakautoindent, bgcolor=bg, fontsize=\small,
  % tabsize=4, breaklines=true, framesep=10pt, frame=leftline, breakindent=2em,
}

% custom commands: optional and voluntary sections
\newcommand{\opcional}{\paragraph{Apartat opcional.}}
\newcommand{\voluntari}{\paragraph{Apartat voluntari.}}

% custom commands: Git commit, filename
\usepackage{xstring}
\newcommand{\commit}[1]{\href{https://github.com/jmendeth/university-3a-dsbm-lab/commit/#1}{commit \texttt{\StrLeft{#1}{6}}}}
\newcommand{\filename}[1]{\mintinline{text}{#1}} % TODO

% custom commands: C function
\newcommand{\fname}[1]{\mintinline{c}{#1}}

% custom commands: LED pattern
\newcommand{\ledpatternledwidth}{0.5em}
\newcommand{\ledpattern}[4]{%
\tikz[
  baseline=-.3em, scale=.5,
  LED/.style={line width=0.5pt, draw=black, minimum width=.01em, minimum height=.05em, fill=none}, LEDX/.style={LED},
  LEDG/.style={LED}, LEDGX/.style={LEDX, fill=green},
  LEDO/.style={LED}, LEDOX/.style={LEDX, fill=orange},
  LEDR/.style={LED}, LEDRX/.style={LEDX, fill=red},
  LEDB/.style={LED}, LEDBX/.style={LEDX, fill=blue},
]{%
  \path [LEDG#1] (-1em,0) ++(-\ledpatternledwidth/2, -\ledpatternledwidth/2) rectangle ++(\ledpatternledwidth, \ledpatternledwidth);
  \path [LEDO#2] (0,+1em) ++(-\ledpatternledwidth/2, -\ledpatternledwidth/2) rectangle ++(\ledpatternledwidth, \ledpatternledwidth);
  \path [LEDR#3] (+1em,0) ++(-\ledpatternledwidth/2, -\ledpatternledwidth/2) rectangle ++(\ledpatternledwidth, \ledpatternledwidth);
  \path [LEDB#4] (0,-1em) ++(-\ledpatternledwidth/2, -\ledpatternledwidth/2) rectangle ++(\ledpatternledwidth, \ledpatternledwidth);
}}

% add PDF bookmark for index
\let\myTOC\tableofcontents
\renewcommand\tableofcontents{%
  \pdfbookmark[1]{\contentsname}{}
  \myTOC }



\begin{document}

\title{Memòria final DSBM}
\date{5 d'octubre de 2018}
\author{Alba Méndez}

\maketitle

\tableofcontents
\listoffigures
\clearpage


\chapter{Introducció}

Aquesta és la memòria final per al laboratori de DSBM, del quadrimestre de tardor del curs 2018/2019.
El laboratori consisteix en la realització d'una sèrie de pràctiques, on l'estudiant programa una
placa donada. El producte final de cada pràctica és, primàriament, el codi de les funcions que es
demanen per a aquella pràctica incloent un o més programes que en demostrin el bon funcionament.

\paragraph{Estructura de la memòria}

En aquesta memòria s'expliquen els objectius, motivació i desenvolupament de cadascuna de les
pràctiques per part de l'estudiant, així com l'assoliment dels objectius especificats i qualsevol
altra nota que es cregui convenient.

S'assumeix que el lector ja està familiaritzat amb el funcionament, perifèrics i especificacions
de la placa que s'empra en el laboratori. S'assumeix també un bon coneixement de l'assignatura.
\textbf{No és necessari} conèixer prèviament el desenvolupament del laboratori per llegir aquesta
memòria; tot i així, de no tenir-lo es recomana llegir cada pràctica en ordre a causa de la natura
incremental d'aquestes.

Cada capítol a continuació descriu la realització d'una pràctica. Hi ha dues pràctiques no
obligatòries en aquesta assignatura, la pràctica~C2 (pàg.~\pageref{ch:c2}) i la pràctica~C3
(pàg.~\pageref{ch:c3}). A part d'això, hi ha «pràctiques voluntàries» fetes espontàneament per
l'estudiant, on aquest ha marcat els objectius a assolir. Aquestes son el selector de programa
(pàg.~\pageref{ch:selector}) i el conversor de base (pàg.~\pageref{ch:basecvt}).

\paragraph{Apartats opcionals i apartats voluntaris}

Dins el desenvolupament de cada pràctica, s'hi han anotat apartats opcionals i apartats voluntaris
fets per l'estudiant. El primer tipus es descriu al manual de pràctiques però la seva realització
no és obligatòria. El segon tipus ha estat purament iniciativa de l'estudiant.

A més, al final del capítol corresponent a cada pràctica hi ha una secció que detalla
millores rellevants al codi d'aquesta pràctica, fetes un cop acabat el desenvolupament d'aquesta
(per exemple, en pràctiques posteriors).

\paragraph{Sobre el codi}

Les pràctiques es desenvolupen de forma incremental, és a dir, cada pràctica acostuma a
basar-se el codi desenvolupat en les pràctiques anteriors per al seu funcionament. Per tant, s'ha
entregat l'estat final del codi que integra el producte de totes les pràctiques.

A més, les pràctiques es descriuen en l'ordre en que apareixen al manual, però l'estudiant no
ha seguit aquest ordre estrictament. Per tant, és possible que s'hagin comés algunes
inconsistències menors entre el codi que es llista en aquesta memòria, i l'entrega del
codi final. Si és necessari es pot consultar el repositori de codi al següent enllaç:

\url{https://github.com/jmendeth/university-3a-dsbm-lab}

On es pot veure, no només l'estat final del codi, sino tot el desenvolupament intermig.

Els fitxers s'entreguen sempre en codificació estàndard UTF-8.
A més, l'estudiant ha escollit indentació de quatre espais per al codi, amb les claus d'obertura
en la mateixa línia. Tots els fitxers donats es formategen previament en aquest format.

\textbf{Important:} En el codi es fan servir algunes extensions de GCC, com ara els
comentaris de línia (\mintinline{c}|// Comment|) o els literals en binari (\mintinline{c}|0b10011|).
Cal destacar també que per motius d'expertesa, l'estudiant ha preferit fer servir el
\href{http://netbeans.org}{NetBeans IDE} per a treballar amb el codi, en comptes d'Eclipse,
que és el que s'utilitza en l'entorn que es dona.

\paragraph{Estudis previs}

Abans de les pràctiques es demanen uns estudis previs a l'estudiant, que s'han inclós en l'entrega.
Aquests també es reprodueixen en la memòria (marcant-ho on pertoqui).


\newcommand{\projectname}{}
\newcommand{\inputproject}[1]{
  \dumpaccum
  \renewcommand{\projectname}{#1}
  \documentclass[catalan,parskip=half*,oneside,DIV=11,hidelinks]{scrreprt}

% encoding
\usepackage[utf8]{inputenc}
\usepackage[T1]{fontenc}
\usepackage{lmodern}
\usepackage{babel}

% formatting and fixes
\frenchspacing
\usepackage[style=spanish]{csquotes}
\MakeAutoQuote{«}{»}
\usepackage{bookmark}
\usepackage{scrhack}

% ADD ANY SPECIFIC PACKAGES HERE
% (CHEMISTRY, CODE, PUBLISHING)
\usepackage{multicol}
\usepackage{environ}
\usepackage{enumitem}
\usepackage[usenames,dvipsnames,svgnames,table]{xcolor}
\usepackage{tcolorbox}
\usepackage{adjustbox}
\usepackage{booktabs}
\usepackage[figure]{hypcap}

\usepackage{mathtools}
\usepackage{commath}
\usepackage{nicefrac}
\usepackage{siunitx}

\usepackage[american]{circuitikz}
\usetikzlibrary{calc}
\usetikzlibrary{arrows.meta}
\usetikzlibrary{automata}
\usepackage{tikz-timing}
\usepackage{minted}

% hyperlink setup / metadata
\usepackage{hyperref}
\AfterPreamble{\hypersetup{
  pdfauthor={Alba Méndez},
  pdftitle={Memòria final DSBM},
  pdfsubject={DSBM QT2018},
  pdfcreator={}, pdfproducer={},
  linkcolor={red!50!black},
  citecolor={blue!50!black},
  urlcolor={blue!80!black},
  bookmarksopen=true,
}}
\setlength{\footskip}{2.0cm}

% custom commands: BDF schematic and BSF symbol
\newcommand{\nodenamebit}[1]{\text{#1}}
\newcommand{\nodenamesingle}[2]{\text{#1}_{#2}}
\newcommand{\nodenamerange}[3]{\text{#1}_{#2..#3}}

% setup C listing appearance
\definecolor{bg}{rgb}{0.95,0.95,0.95}
\usemintedstyle{lovelace}
\setminted{
  breaklines, breakindent=1em, breakautoindent, bgcolor=bg, fontsize=\small,
  % tabsize=4, breaklines=true, framesep=10pt, frame=leftline, breakindent=2em,
}

% custom commands: optional and voluntary sections
\newcommand{\opcional}{\paragraph{Apartat opcional.}}
\newcommand{\voluntari}{\paragraph{Apartat voluntari.}}

% custom commands: Git commit, filename
\usepackage{xstring}
\newcommand{\commit}[1]{commit \href{https://github.com/jmendeth/university-3a-dsbm-lab/commit/#1}{\texttt{\StrLeft{#1}{6}}}}
\newcommand{\filename}[1]{\mintinline{text}{#1}} % TODO

% custom commands: C function
\newcommand{\fname}[1]{\mintinline{c}{#1}}

% custom commands: LED pattern
\newcommand{\ledpatternledwidth}{0.5em}
\newcommand{\ledpattern}[4]{%
\tikz[
  baseline=-.3em, scale=.5,
  LED/.style={line width=0.5pt, draw=black, minimum width=.01em, minimum height=.05em, fill=none}, LEDX/.style={LED},
  LEDG/.style={LED}, LEDGX/.style={LEDX, fill=green},
  LEDO/.style={LED}, LEDOX/.style={LEDX, fill=orange},
  LEDR/.style={LED}, LEDRX/.style={LEDX, fill=red},
  LEDB/.style={LED}, LEDBX/.style={LEDX, fill=blue},
]{%
  \path [LEDG#1] (-1em,0) ++(-\ledpatternledwidth/2, -\ledpatternledwidth/2) rectangle ++(\ledpatternledwidth, \ledpatternledwidth);
  \path [LEDO#2] (0,+1em) ++(-\ledpatternledwidth/2, -\ledpatternledwidth/2) rectangle ++(\ledpatternledwidth, \ledpatternledwidth);
  \path [LEDR#3] (+1em,0) ++(-\ledpatternledwidth/2, -\ledpatternledwidth/2) rectangle ++(\ledpatternledwidth, \ledpatternledwidth);
  \path [LEDB#4] (0,-1em) ++(-\ledpatternledwidth/2, -\ledpatternledwidth/2) rectangle ++(\ledpatternledwidth, \ledpatternledwidth);
}}

% add PDF bookmark for index
\let\myTOC\tableofcontents
\renewcommand\tableofcontents{%
  \pdfbookmark[1]{\contentsname}{}
  \myTOC }



\begin{document}

\title{Memòria final DSBM}
\date{5 d'octubre de 2018}
\author{Alba Méndez}

\maketitle

\tableofcontents
\clearpage


\newcommand{\projectname}{}
\newcommand{\inputproject}[1]{
  \dumpaccum
  \renewcommand{\projectname}{#1}
  \documentclass[catalan,parskip=half*,oneside,DIV=11,hidelinks]{scrreprt}

% encoding
\usepackage[utf8]{inputenc}
\usepackage[T1]{fontenc}
\usepackage{lmodern}
\usepackage{babel}

% formatting and fixes
\frenchspacing
\usepackage[style=spanish]{csquotes}
\MakeAutoQuote{«}{»}
\usepackage{bookmark}
\usepackage{scrhack}

% ADD ANY SPECIFIC PACKAGES HERE
% (CHEMISTRY, CODE, PUBLISHING)
\usepackage{multicol}
\usepackage{environ}
\usepackage{enumitem}
\usepackage[usenames,dvipsnames,svgnames,table]{xcolor}
\usepackage{tcolorbox}
\usepackage{adjustbox}
\usepackage{booktabs}
\usepackage[figure]{hypcap}

\usepackage{mathtools}
\usepackage{commath}
\usepackage{nicefrac}
\usepackage{siunitx}

\usepackage[american]{circuitikz}
\usetikzlibrary{calc}
\usetikzlibrary{arrows.meta}
\usetikzlibrary{automata}
\usepackage{tikz-timing}
\usepackage{minted}

% hyperlink setup / metadata
\usepackage{hyperref}
\AfterPreamble{\hypersetup{
  pdfauthor={Alba Méndez},
  pdftitle={Memòria final DSBM},
  pdfsubject={DSBM QT2018},
  pdfcreator={}, pdfproducer={},
  linkcolor={red!50!black},
  citecolor={blue!50!black},
  urlcolor={blue!80!black},
  bookmarksopen=true,
}}
\setlength{\footskip}{2.0cm}

% custom commands: BDF schematic and BSF symbol
\newcommand{\nodenamebit}[1]{\text{#1}}
\newcommand{\nodenamesingle}[2]{\text{#1}_{#2}}
\newcommand{\nodenamerange}[3]{\text{#1}_{#2..#3}}

% setup C listing appearance
\definecolor{bg}{rgb}{0.95,0.95,0.95}
\usemintedstyle{lovelace}
\setminted{
  breaklines, breakindent=1em, breakautoindent, bgcolor=bg, fontsize=\small,
  % tabsize=4, breaklines=true, framesep=10pt, frame=leftline, breakindent=2em,
}

% custom commands: optional and voluntary sections
\newcommand{\opcional}{\paragraph{Apartat opcional.}}
\newcommand{\voluntari}{\paragraph{Apartat voluntari.}}

% custom commands: Git commit, filename
\usepackage{xstring}
\newcommand{\commit}[1]{commit \href{https://github.com/jmendeth/university-3a-dsbm-lab/commit/#1}{\texttt{\StrLeft{#1}{6}}}}
\newcommand{\filename}[1]{\mintinline{text}{#1}} % TODO

% custom commands: C function
\newcommand{\fname}[1]{\mintinline{c}{#1}}

% custom commands: LED pattern
\newcommand{\ledpatternledwidth}{0.5em}
\newcommand{\ledpattern}[4]{%
\tikz[
  baseline=-.3em, scale=.5,
  LED/.style={line width=0.5pt, draw=black, minimum width=.01em, minimum height=.05em, fill=none}, LEDX/.style={LED},
  LEDG/.style={LED}, LEDGX/.style={LEDX, fill=green},
  LEDO/.style={LED}, LEDOX/.style={LEDX, fill=orange},
  LEDR/.style={LED}, LEDRX/.style={LEDX, fill=red},
  LEDB/.style={LED}, LEDBX/.style={LEDX, fill=blue},
]{%
  \path [LEDG#1] (-1em,0) ++(-\ledpatternledwidth/2, -\ledpatternledwidth/2) rectangle ++(\ledpatternledwidth, \ledpatternledwidth);
  \path [LEDO#2] (0,+1em) ++(-\ledpatternledwidth/2, -\ledpatternledwidth/2) rectangle ++(\ledpatternledwidth, \ledpatternledwidth);
  \path [LEDR#3] (+1em,0) ++(-\ledpatternledwidth/2, -\ledpatternledwidth/2) rectangle ++(\ledpatternledwidth, \ledpatternledwidth);
  \path [LEDB#4] (0,-1em) ++(-\ledpatternledwidth/2, -\ledpatternledwidth/2) rectangle ++(\ledpatternledwidth, \ledpatternledwidth);
}}

% add PDF bookmark for index
\let\myTOC\tableofcontents
\renewcommand\tableofcontents{%
  \pdfbookmark[1]{\contentsname}{}
  \myTOC }



\begin{document}

\title{Memòria final DSBM}
\date{5 d'octubre de 2018}
\author{Alba Méndez}

\maketitle

\tableofcontents
\clearpage


\newcommand{\projectname}{}
\newcommand{\inputproject}[1]{
  \dumpaccum
  \renewcommand{\projectname}{#1}
  \documentclass[catalan,parskip=half*,oneside,DIV=11,hidelinks]{scrreprt}

% encoding
\usepackage[utf8]{inputenc}
\usepackage[T1]{fontenc}
\usepackage{lmodern}
\usepackage{babel}

% formatting and fixes
\frenchspacing
\usepackage[style=spanish]{csquotes}
\MakeAutoQuote{«}{»}
\usepackage{bookmark}
\usepackage{scrhack}

% ADD ANY SPECIFIC PACKAGES HERE
% (CHEMISTRY, CODE, PUBLISHING)
\usepackage{multicol}
\usepackage{environ}
\usepackage{enumitem}
\usepackage[usenames,dvipsnames,svgnames,table]{xcolor}
\usepackage{tcolorbox}
\usepackage{adjustbox}
\usepackage{booktabs}
\usepackage[figure]{hypcap}

\usepackage{mathtools}
\usepackage{commath}
\usepackage{nicefrac}
\usepackage{siunitx}

\usepackage[american]{circuitikz}
\usetikzlibrary{calc}
\usetikzlibrary{arrows.meta}
\usetikzlibrary{automata}
\usepackage{tikz-timing}
\usepackage{minted}

% hyperlink setup / metadata
\usepackage{hyperref}
\AfterPreamble{\hypersetup{
  pdfauthor={Alba Méndez},
  pdftitle={Memòria final DSBM},
  pdfsubject={DSBM QT2018},
  pdfcreator={}, pdfproducer={},
  linkcolor={red!50!black},
  citecolor={blue!50!black},
  urlcolor={blue!80!black},
  bookmarksopen=true,
}}
\setlength{\footskip}{2.0cm}

% custom commands: BDF schematic and BSF symbol
\newcommand{\nodenamebit}[1]{\text{#1}}
\newcommand{\nodenamesingle}[2]{\text{#1}_{#2}}
\newcommand{\nodenamerange}[3]{\text{#1}_{#2..#3}}

% setup C listing appearance
\definecolor{bg}{rgb}{0.95,0.95,0.95}
\usemintedstyle{lovelace}
\setminted{
  breaklines, breakindent=1em, breakautoindent, bgcolor=bg, fontsize=\small,
  % tabsize=4, breaklines=true, framesep=10pt, frame=leftline, breakindent=2em,
}

% custom commands: optional and voluntary sections
\newcommand{\opcional}{\paragraph{Apartat opcional.}}
\newcommand{\voluntari}{\paragraph{Apartat voluntari.}}

% custom commands: Git commit, filename
\usepackage{xstring}
\newcommand{\commit}[1]{commit \href{https://github.com/jmendeth/university-3a-dsbm-lab/commit/#1}{\texttt{\StrLeft{#1}{6}}}}
\newcommand{\filename}[1]{\mintinline{text}{#1}} % TODO

% custom commands: C function
\newcommand{\fname}[1]{\mintinline{c}{#1}}

% custom commands: LED pattern
\newcommand{\ledpatternledwidth}{0.5em}
\newcommand{\ledpattern}[4]{%
\tikz[
  baseline=-.3em, scale=.5,
  LED/.style={line width=0.5pt, draw=black, minimum width=.01em, minimum height=.05em, fill=none}, LEDX/.style={LED},
  LEDG/.style={LED}, LEDGX/.style={LEDX, fill=green},
  LEDO/.style={LED}, LEDOX/.style={LEDX, fill=orange},
  LEDR/.style={LED}, LEDRX/.style={LEDX, fill=red},
  LEDB/.style={LED}, LEDBX/.style={LEDX, fill=blue},
]{%
  \path [LEDG#1] (-1em,0) ++(-\ledpatternledwidth/2, -\ledpatternledwidth/2) rectangle ++(\ledpatternledwidth, \ledpatternledwidth);
  \path [LEDO#2] (0,+1em) ++(-\ledpatternledwidth/2, -\ledpatternledwidth/2) rectangle ++(\ledpatternledwidth, \ledpatternledwidth);
  \path [LEDR#3] (+1em,0) ++(-\ledpatternledwidth/2, -\ledpatternledwidth/2) rectangle ++(\ledpatternledwidth, \ledpatternledwidth);
  \path [LEDB#4] (0,-1em) ++(-\ledpatternledwidth/2, -\ledpatternledwidth/2) rectangle ++(\ledpatternledwidth, \ledpatternledwidth);
}}

% add PDF bookmark for index
\let\myTOC\tableofcontents
\renewcommand\tableofcontents{%
  \pdfbookmark[1]{\contentsname}{}
  \myTOC }



\begin{document}

\title{Memòria final DSBM}
\date{5 d'octubre de 2018}
\author{Alba Méndez}

\maketitle

\tableofcontents
\clearpage


\newcommand{\projectname}{}
\newcommand{\inputproject}[1]{
  \dumpaccum
  \renewcommand{\projectname}{#1}
  \input{../#1/main.tex}
}
%\inputproject{P1}

\end{document}

}
%\inputproject{P1}

\end{document}

}
%\inputproject{P1}

\end{document}

}
%\inputproject{P1}

\end{document}
