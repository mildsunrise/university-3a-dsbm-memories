\projectstart{r2}{R2}{Compartició d'elements comuns}

\section{Objectius}

En aquesta pràctica es vol fer una introducció a dues eines bàsiques
de sincronització entre processos: els semàfors binaris i els mutexes.
S'exposaran les utilitats i la forma d'us de cadascuna, i es treballarà
sobre dos programes d'exemple per demostrar-ho.

\section{Desenvolupament}


\subsection{Introducció als Semàfors}

S'explica la necessitat de protegir l'accés a elements comuns de forma que
només hi pugui accedir un procés alhora. S'introdueixen els \emph{semàfors binaris}
de ChibiOS/RT com a possible solució. Llavors s'explica l'us bàsic d'un semàfor.


\subsection{Exemple d'ús de Semàfors}

En primer lloc es demana importar al projecte els fitxers \filename{mutexes.c} i \filename{mutexes.h}
i incloure el primer al \filename{Makefile}. Això es fa al \commit{99b853f483a6605d7cde3d9ae6dfabf453bb6a74}.

Ara treballarem amb la funció \fname{semaphoreExample}. El programa és el següent:

\begin{minted}{c}
// Binary semaphore
BinarySemaphore semaf;

// Working area for the semaphore child thread
static WORKING_AREA(waSem, 128);

// Child thread function prototype
static msg_t thSem(void *arg);

// Process syncronization example that uses semaphores

void semaphoreExample(void) {
    // Global initialization
    baseInit();

    // Initializes the semaphore as not taken
    chBSemInit(&semaf, FALSE);

    // Creates a child thread
    chThdCreateStatic(waSem, sizeof (waSem), NORMALPRIO + 1, thSem, NULL);

    while (1) {
        SLEEP_MS(300);          // Wait 300ms
        chBSemSignal(&semaf);   // Send signal to child
    }
}


// Child thread that changes the state of the orange LED
// at each semaphore syncronization

static msg_t thSem(void *arg) {
    while (1) {
        chBSemWait(&semaf);                 // Wait for syncronization
        (LEDS_PORT->ODR) ^= ORANGE_LED_BIT; // Toggle LED
    }

    return 0;
}
\end{minted}
\vskip -1em

Com podem veure, el programa comença inicialitzant el sistema i el semàfor, i creant un nou procés fill.
Fet això, es limita a alliberar el semàfor cada \SI{300}{\milli\second},

Per altra banda, el procés fill intenta repetidament agafar el semàfor, i cada cop que això passa
canvia l'estat del LED taronja~\ledpattern{ }{X}{ }{ }. En la primera iteració el podrà agafar inmediatament
ja que inicialment està lliure, però en la segona quedarà bloquejat. Quan després de \SI{300}{\milli\second}
el fil principal alliberi el semàfor, \fname{chBSemWait} el podrà tornar a agafar i es farà una iteració més.

El resultat, doncs, és que el LED canvia d'estat constantment. La funció es crida des del \fname{main}
i es comprova que el funcionament és l'esperat.

Ara es demana, com a estudi previ, modificar el programa perquè creï un tercer procés \fname{thSem2}
de la mateixa prioritat que \fname{thSem} i que estarà controlat per un segon semàfor \mintinline{c}|semaf2|
i operarà sobre el LED blau~\ledpattern{ }{ }{ }{X}. Llavors, en comptes de desbloquejar el semàfor
en el procés principal, es desbloquejarà cada dues iteracions de \fname{thSem}.

%previ
En primer lloc, es defineix i s'inicialitza el semàfor \mintinline{c}|semaf2|:

\begin{minted}{diff}
@@ -13,7 +13,7 @@
 /******************* SEMAPHORE EXAMPLE ********************/
 
 // Binary semaphores
-BinarySemaphore semaf;
+BinarySemaphore semaf, semaf2;
 
 // Working area for the semaphore child thread
 static WORKING_AREA(waSem, 128);
@@ -27,7 +28,8 @@ void semaphoreExample(void) {
     baseInit();
 
     // Initializes the semaphores as not taken
     chBSemInit(&semaf, FALSE);
+    chBSemInit(&semaf2, FALSE);
 
     // Creates a child thread
     chThdCreateStatic(waSem, sizeof (waSem), NORMALPRIO + 1, thSem, NULL);
\end{minted}
\vskip -1em

En segon lloc, fem una còpia de \fname{thSem} i la canviem perquè
que controli el LED blau i esperi a \mintinline{c}|semaf2|:

\begin{minted}{c}
// Variation of the original thSem but controlling the blue LED
// and synchronized on the second semaphore

static msg_t thSem2(void *arg) {
    while (1) {
        chBSemWait(&semaf2);                // Wait for synchronization
        (LEDS_PORT->ODR) ^= BLUE_LED_BIT;   // Toggle LED
    }

    return 0;
}
\end{minted}
\vskip -1em

En tercer lloc, només queda modificar \fname{thSem} perquè
desbloquegi \mintinline{c}|semaf2| cada dos canvis de LED:

\begin{minted}{diff}
 static msg_t thSem(void *arg) {
+    uint32_t counter = 0;
     while (1) {
         chBSemWait(&semaf);                 // Wait for syncronization
         (LEDS_PORT->ODR) ^= ORANGE_LED_BIT; // Toggle LED
 
+        counter++;
+        if ((counter % 2) == 0)             // Every two ticks
+            chBSemSignal(&semaf2);          // Send signal to other child
     }
 
     return 0;
 }
\end{minted}
\vskip -1em

I en últim lloc, només queda arrencar el segón procés a la funció
principal (en el codi final, aquestes modificacions s'han fet sobre una
còpia de la funció anomenada \fname{semaphoreTwoThreads}):

\begin{minted}{diff}
@@ -16,7 +16,8 @@
 BinarySemaphore semaf, semaf2;
 
 // Working area for the semaphore child threads
 static WORKING_AREA(waSem, 128);
+static WORKING_AREA(waSem2, 128);
 
 // Child thread function prototype
 static msg_t thSem(void *arg);
@@ -32,8 +32,9 @@ void semaphoreExample(void) {
     chBSemInit(&semaf2, FALSE);
 
     // Creates a child thread
     chThdCreateStatic(waSem, sizeof (waSem), NORMALPRIO + 1, thSem, NULL);
+    chThdCreateStatic(waSem2, sizeof (waSem2), NORMALPRIO + 1, thSem2, NULL);
 
     while (1) {
         SLEEP_MS(300);          // Wait 300ms
\end{minted}
\vskip -1em
%/previ

Cridem el programa des del \fname{main} i el carreguem a la placa. El resultat
és l'esperat: el LED blau fa un canvi d'estat cada dos canvis d'estat del LED
taronja (per tant va la meitat de ràpid).
Això conforma el \commit{81f976c6f794a8b2512e715932d0f71bd8745259}.


\subsection{Control d'elements comuns mitjançant Mutexes}

En aquesta segona part de la pràctica es vol introduir el concepte de \emph{mutexes}.

Primer s'explica com sincronitzariem l'accés a un recurs comú (en el nostre cas, l'LCD)
i per què són més recomanables els mutexes que els semàfors per a tasques com aquesta.
Es parla dels \emph{deadlocks}, del fenòmen d'inversió de prioritat i de l'heretament
de prioritat de què disposa ChibiOS/RT. S'explica el seu us més bàsic.


\subsection{Exemple d'ús de Mutexes}

Ara treballarem sobre la segona part del fitxer, que conté el següent programa
d'exemple (funció \fname{mutexExample}):

\begin{minted}{c}
// Working area for the child thread
static WORKING_AREA(waThEM, 128);

// Child thread function prototype
static msg_t thMtx(void *arg);

// Test where two threads access to the LCD
// The main thread write digits 0..9 on the first LCD line
// The child thread write letters A..Z on the second LCD line

void mutexExample(void) {
    int32_t x, i, car;

    // Global initialization
    baseInit();

    // Initialize the LCD
    LCD_Init();

    // Clear the LCD and turn off the backlight
    LCD_ClearDisplay();
    LCD_Backlight(0);

    // Creates a child thread
    chThdCreateStatic(waThEM, sizeof (waThEM), NORMALPRIO, thMtx, NULL);

    // First digit to show
    car = '0';

    // Infinite loop
    while (1) {
        for (i = 0; i < 20; i++)    // 20 times for each digit
            for (x = 0; x < LCD_COLUMNS; x++) { // For each column on the LCD...
                LCD_GotoXY(x, 0);       // Jump to that column
                LCD_SendChar(car);      // Write the digit
                DELAY_US(17000);        // Some delay
            }
        if (++car > '9') car = '0'; // Go to next digit
    }
}

// Child thread entry point
// Write letters on the second row of the LCD

static msg_t thMtx(void *arg) {
    int32_t x, i, car;

    // First char to show
    car = 'A';

    // Infinite loop
    while (1) {
        for (i = 0; i < 20; i++)    // 20 times for each char
            for (x = 0; x < LCD_COLUMNS; x++) { // For each LCD column
                LCD_GotoXY(x, 1);       // Jump to that column
                LCD_SendChar(car);      // Write the char
                DELAY_US2(10000);       // Some delay
            }
        if (++car > 'Z') car = 'A'; // Go to next char
    }
    return 0;
}
\end{minted}
\vskip -1em

Com podem veure, es tracta d'un programa que inicia un procés fill, i tots dos
processos escriuen periòdicament dígits a diferents posicions de l'LCD.
Com que fan servir \fname{DELAY_US} i no \fname{SLEEP_MS}, no cedeixen temps
de CPU i l'scheduler va intercanviant entre un i l'altre (round-robin).

Cridem \fname{mutexExample} des de \fname{main} i ho carreguem a la placa,
i efectivament, l'accés concurrent a l'LCD acaba provocant glitches i
finalment la pantalla falla completament.

Per tant, volem arbitrar l'accés a l'LCD fent servir un mutex que tots
dos processos hauran d'agafar abans de fer-lo servir, i alliberar-lo després.

Es demana (estudi previ) implementar les modificacions al programa. Cal
definir el mutex, inicialitzar-lo, i fer-lo servir en les dues àrees d'accés
a l'LCD:

%previ
\begin{minted}{diff}
@@ -90,6 +90,9 @@ void semaphoreTwoThreads(void) {
 
 /************************ MUTEX EXAMPLE *******************************/
 
+// Mutex protecting LCD access
+static Mutex mutex;
+
 // Working area for the child thread
 static WORKING_AREA(waThEM, 128);
 
@@ -113,6 +116,9 @@ void mutexExample(void) {
     LCD_ClearDisplay();
     LCD_Backlight(0);
 
+    // Initialize the mutex
+    chMtxInit(&mutex);
+
     // Creates a child thread
     chThdCreateStatic(waThEM, sizeof (waThEM), NORMALPRIO, thMtx, NULL);
 
@@ -123,8 +129,10 @@ void mutexExample(void) {
     while (1) {
         for (i = 0; i < 20; i++)    // 20 times for each digit
             for (x = 0; x < LCD_COLUMNS; x++) { // For each column on the LCD...
+                chMtxLock(&mutex);      // LOCK the LCD mutex
                 LCD_GotoXY(x, 0);       // Jump to that column
                 LCD_SendChar(car);      // Write the digit
+                chMtxUnlock();          // UNLOCK the LCD mutex
                 DELAY_US(17000);        // Some delay
             }
         if (++car > '9') car = '0'; // Go to next digit
@@ -144,8 +152,10 @@ static msg_t thMtx(void *arg) {
     while (1) {
         for (i = 0; i < 20; i++)    // 20 times for each char
             for (x = 0; x < LCD_COLUMNS; x++) { // For each LCD column
+                chMtxLock(&mutex);      // LOCK the LCD mutex
                 LCD_GotoXY(x, 1);       // Jump to that column
                 LCD_SendChar(car);      // Write the char
+                chMtxUnlock();          // UNLOCK the LCD mutex
                 DELAY_US2(10000);       // Some delay
             }
         if (++car > 'Z') car = 'A'; // Go to next char
\end{minted}
\vskip -1em
%/previ

Es carrega a la placa de nou, i comprovem que ara ja no apareixen glitches
a l'LCD. Això conforma el \commit{d1253453f176e8d3ca27002cd4cf1314c6677577}.


\section{Conclusió}

Aquesta pràctica s'ha realitzat sense problemes.

\section{Ajustaments posteriors}

Cap canvi a destacar.
% TODO
