\projectstart{r1}{R1}{Introducció a ChibiOS/RT}

\section{Objectius}

En l'entorn de desenvolupament que se'ns dona ja fet, el codi es compila amb
ChibiOS/RT, un RTOS. Tot i que fins ara es podria dir que no n'hem
fet us, en aquesta pràctica s'explicarà la necessitat i els beneficis de fer-lo
servir, el funcionament dels fils o processos, i el de la capa d'abstracció del
hardware.

Finalment, es faran petits programes per fer us dels fils.

\section{Desenvolupament}


\subsection{ChibiOS/RT}

Es comença explicant què és un sistema operatiu, la necessitat de fer-ne servir un quan
l'aplicació que programem comença a ser més complexa, i la comoditat que ens dona des del
punt de vista de la programació els processos/fils.

També s'explica què és la capa d'abstracció del hardware, que de vegades pot venir inclosa
en el sistema operatiu, i els beneficis de fer-la servir (portabilitat, protegir l'accés
concurrent al hardware).

Llavors s'entra concretament en els sistemes operatius en temps real (RTOS, el nostre cas).
S'explica què és el requeriment de temps real que pot tenir un sistema, i per què necessitem
un RTOS per complir-lo, i no qualsevol tipus de sistema operatiu.

S'introdueix llavors ChibiOS/RT, un RTOS de codi obert amb que es compila el nostre codi en
l'entorn de desenvolupament donat. S'explica que fins la versió que fem servir té integrat
el HAL dins el sistema operatiu, es detallen la resta de parts (kernel, port) i la funció
de cadascuna.

\subsection{Processos}

S'explica què és l'\emph{scheduler} del kernel, que la seva funció és repartir el temps entre
processos, i la política que fa servir: es basa en prioritat i \emph{round robin} dins una
mateixa prioritat. S'explica què és el tic de scheduler, què és el \emph{time quantum} i quant
valen per defecte. També es recalca el cost de fer un canvi de context.

Llavors s'explica què hi ha en la memòria que reservem per a un procés abans de crear-lo;
la seva pila (\emph{stack}), informació de context i informació del procés que fa servir el
SO per identificar-lo.

\subsection{Exemple de codi}

En primer lloc es demana importar al projecte els fitxers \filename{process.c} i \filename{process.h}
i incloure el primer al \filename{Makefile}. Això es fa al \commit{99b853f483a6605d7cde3d9ae6dfabf453bb6a74}.

Ara treballarem amb la funció \fname{test2threads}, que implementa un exemple senzill amb dos processos.
El procés principal crea un procés fill i llavors fa pampallugues al LED blau~\ledpattern{ }{ }{ }{X}
per indicar que està corrent. El procés fill, que té la mateixa prioritat que el principal,
per la seva banda fa pampallugues amb el LED taronja~\ledpattern{ }{X}{ }{ } un cop engegat:

\begin{minted}{c}
// Working area for the child thread
static WORKING_AREA(waChild, 128);

// Function prototype for the child thread
// Needed because thChild is referenced before its definition
// It is not included in process.h because it's a static function
static msg_t thChild(void *arg);

// Test for simple two thread operation
// This function:
//     - Initializes the system
//     - Creates a child thread that blinks the orange LED
//     - Blinks the blue LED

void test2threads(void) {
    // Basic system initialization
    baseInit();

    // Child thread creation
    chThdCreateStatic(waChild, sizeof (waChild), NORMALPRIO, thChild, NULL);

    while (TRUE) {
        // Turn on blue LED using BSRR
        (LEDS_PORT->BSRR.H.set) = BLUE_LED_BIT;

        // Pause
        busyWait(3);

        // Turn off blue LED using BSRR
        (LEDS_PORT->BSRR.H.clear) = BLUE_LED_BIT;

        // Pausa
        busyWait(3);
    }
}


// Child thread that bliks the orange LED

static msg_t thChild(void *arg) {
    while (TRUE) {
        // Turn on orange LED using BSRR
        (LEDS_PORT->BSRR.H.set) = ORANGE_LED_BIT;

        // Pausa
        busyWait(5);

        // Turn off orange LED using BSRR
        (LEDS_PORT->BSRR.H.clear) = ORANGE_LED_BIT;

        // Pausa
        busyWait(5);
    }
    return 0;
}

// Busy waits in a thread doing some operations
// The greater n, the longer the wait

void busyWait(uint32_t n) {
    uint32_t i;
    volatile uint32_t x = 0;
    for (i = 0; i < n * 1000000; i++)
        x = (x + 2) / 3;
}
\end{minted}
\vskip -1em

S'explica en profunditat cada part del codi, i es demana que cridem a la funció \fname{test2threads}
des de \fname{main}. Es comprova que tots dos processos, al tenir la mateixa prioritat, corren alhora
(intercanviant-se cada \emph{quantum}).

Ara es demana que canviem la prioritat del procés fill, de \mintinline{c}|NORMALPRIO| a
\mintinline{c}|NORMALPRIO+1|, passant a tenir prioritat superior al procés principal.
Per tal de mantenir el codi net i reproduïble, es fa una còpia de la funció que anomenem
\fname{test2threadsPlus1} i fem allà el canvi. La cridem des de \fname{main}, carreguem el
programa a la placa i verifiquem que, tal i com s'espera, només corre el procés fill ja que
és el que té més prioritat. Per tant només fa pampallugues el LED taronja.

També es demana que canviem la prioritat a \mintinline{c}|NORMALPRIO-1|.
Anomenem la nova funció, \fname{test2threadsMinus1}. La carreguem i efectivament,
ara només s'executa el procés principal ja que és el de més prioritat, i per tant
només fa pampallugues el LED blau.

A continuació s'exposa la mala pràctica de fer treballar la CPU en estats d'espera,
i és proposa que canviem les crides a \mintinline{c}|busyWait(3)| per crides a \mintinline{c}|SLEEP_MS(300)|,
i de forma similar amb \mintinline{c}|busyWait(5)|. Això permet que altres processos
facin servir la CPU mentre el procés actual espera.

De nou fem una còpia amb els canvis de dalt, i creant el fill amb
prioritat \mintinline{c}|NORMALPRIO+1|. Anomenem la funció \fname{test2threadsSleep},
la pugem i comprovem que efectivament, encara que el fill té més prioritat que el pare,
quan el primer entra en espera el segon s'executa i per tant tots dos LEDs fan
pampallugues alhora.

El que hem vist és una forma d'un procés per bloquejar-se a sí mateix durant un periode
de temps. S'expliquen altres formes de bloquejar-se, com ara esperar a una interrupció,
a que un element d'us comú quedi lliure, o a que un altre procés acabi.

Per últim, es demana ampliar el programa per tal de tenir 4 processos totals, cadascun
fent pampallugues en un dels quatre LEDs i amb diferents esperes.

\opcional
Per simplificar el codi, farem una única funció genèrica que agafi uns paràmetres (bit del LED
a controlar, i temps d'espera) i implementi el bucle infinit per fer pampallugues. S'engegen tres
fils amb aquesta funció i diferents paràmetres, i al fil principal simplement es crida
amb altres paràmetres.

De nou, per mantenir el codi net farem el programa independent de la resta. El codi
implementat (estudi previ) és:

%previ
\begin{minted}{c}
// Input parameters struct for the thread function
typedef struct { uint32_t bits; int32_t sleepTime; } FourThreadsChildParams;

// Thread function
static msg_t fourThreadsChild(void *arg) {
    FourThreadsChildParams *params = (FourThreadsChildParams*) arg;
    while (1) {
        LEDS_PORT->BSRR.H.set = params->bits;
        SLEEP_MS(params->sleepTime);
        LEDS_PORT->BSRR.H.clear = params->bits;
        SLEEP_MS(params->sleepTime);
    }
    return 0;
}

// Main program
void fourThreadsTest(void) {
    // Allocate working space for each child thread
    static struct { WORKING_AREA(wa, 128); } children [3];

    // Define parameters for each thread
    FourThreadsChildParams params [] = {
        { BLUE_LED_BIT, 300 },
        { ORANGE_LED_BIT, 500 },
        { GREEN_LED_BIT, 400 },
        { RED_LED_BIT, 700 },
    };

    // Start child threads
    int32_t i;
    for (i = 0; i < 3; i++)
        chThdCreateStatic(children[i].wa, sizeof(children[i].wa), NORMALPRIO, fourThreadsChild, &params[i+1]);

    // Start main thread's routine
    fourThreadsChild(&params[0]);
}
\end{minted}
\vskip -1em
%/previ

El codi s'inserta a \filename{main.c}, es crida des de \fname{main}, es carrega a la placa
i es comprova el funcionament esperat (tots els processos corrent alhora, tots els LEDs fent
pampallugues amb diferents esperes). Tots aquests programes conformen
el \commit{d84e6148aa3ea8623b9c51ef6ced18b39583bfd8}.


\section{Conclusió}

Cap problema rellevant. Considero que s'ha fet un bon treball explicant tots els conceptes
necessaris i el programa demanat al final dona una bona iniciació als fils.

\section{Ajustaments posteriors}

Cap canvi a destacar.
% TODO
